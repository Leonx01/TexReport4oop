\documentclass{article}
\usepackage{listings} 
\usepackage{xcolor}
\usepackage{color}
\usepackage{xcolor}
\usepackage{geometry}
\usepackage{ctex}
\usepackage{ulem}
\usepackage{graphicx}
\usepackage{CJKfntef}
\usepackage{CJKulem}
\usepackage{soul}
\usepackage{fancyhdr}
\usepackage{subfigure}
% 目录字体设置
\usepackage[subfigure]{tocloft}


\renewcommand{\contentsname}{\kaishu\fontsize{24pt}{\baselineskip}\selectfont \textbf{目录}}
\renewcommand{\cftsecleader}{\cftdotfill{0.6}}
\renewcommand{\cftsubsecleader}{\cftdotfill{0.6}}

\renewcommand{\cftsecfont}{\kaishu \fontsize{15pt}{10pt}\selectfont  }
\renewcommand{\cftsubsecfont}{\kaishu\fontsize{10pt}{0pt}\selectfont}


\pagestyle{fancy}
% 页眉设置
%\lhead{\includegraphics[scale = 0.01]{Page.jpg}}
\fancyhead[L]{\includegraphics[width = 3cm,height = 0.6cm]{XJTU.png}}
\fancyhead[R]{面向对象程序设计实验报告}
%\fancyhead[C]{中间页眉}
% 页脚设置
\fancyfoot[C]{\thepage}
\fancyfoot[R]{}
\renewcommand{\headrulewidth}{0.2pt} % 分隔线宽度4磅
\renewcommand{\footrulewidth}{0.2pt}

\definecolor{red}{RGB}{200,30,22}
%自定义交大红色
\newcommand\reduline{\bgroup\markoverwith
{\textcolor{red}{\rule[-0.5ex]{10pt}{3pt}}}\ULon}
%自定义命令
\geometry{left=3.17cm,right=3.17cm,top=1.75cm,bottom=2.54cm}
%设置页边距

\definecolor{dkgreen}{rgb}{0,0.6,0}
\definecolor{gray}{rgb}{0.5,0.5,0.5}
\definecolor{mauve}{rgb}{0.58,0,0.82}
\lstset{frame=tb,
	language=Java, % 使用的语言
	aboveskip=3mm,
	belowskip=3mm,
	showstringspaces=false, % 仅在字符串中允许空格
	backgroundcolor=\color{white},   % 选择代码背景,必须加上\ usepackage {color}或\ usepackage {xcolor}
	columns=flexible,
	basicstyle = \ttfamily\small,
	numberstyle=\small \color{gray},  % 行号的字号和颜色
	numbers=left, %给代码添加行号,可取值none, left, right.
	keywordstyle=\color{blue},
	commentstyle=\color{dkgreen}, % 设置注释格式
	stringstyle=\color{mauve},
	breaklines=true,   % 设置自动断行.
	breakatwhitespace=true, % 设置是否当且仅当在空白处自动中断.
	escapeinside=``, %逃逸字符(1左面的键),用于显示中文t
	frame=single, %设置边框格式
	extendedchars=false, %解决代码跨页时,章节标题,页眉等汉字不显示的问题
	xleftmargin=2em,xrightmargin=2em, aboveskip=1em, %设置边距
	tabsize=4 % 将默认tab设置为4个空格
}

\title{\textbf {\kaishu\fontsize{48pt} {48pt}\selectfont 面向对象程序设计\\[15pt]作业报告}}
\author{\reduline{\fontsize{1pt}{1pt}\selectfont \hspace*{400pt}}}
\date{\fontsize{25pt}{20pt}\selectfont 第\,\,N\,\,次}

\begin{document}
\maketitle
\begin{center}
\includegraphics[scale = 0.14]{honor.jpg}%交大校徽
\\[50pt] %空行
\end{center}

\begin{description}
    \linespread{2}
    \item[\kaishu \fontsize{20pt}{20pt}\selectfont 姓名]\hspace*{50pt} \CJKunderdblline{\kaishu \fontsize{20pt}{20pt}\selectfont XXX \hspace*{250pt}}
    \item[\kaishu \fontsize{20pt}{20pt}\selectfont 班级]\hspace*{50pt}\CJKunderdblline{\kaishu \fontsize{20pt}{20pt}\selectfont 软件xxxx班\hspace*{218pt}}
    \item[\kaishu \fontsize{20pt}{20pt}\selectfont 学号]\hspace*{50pt} \CJKunderdblline{\fontsize{20pt}{20pt}\selectfont  1111111111\hspace*{220pt}}
    \item[\kaishu \fontsize{20pt}{20pt}\selectfont 电话]\hspace*{50pt} \CJKunderdblline {\fontsize{20pt}{20pt}\selectfont  11111111111\hspace*{210pt}}
    \item[\sffamily \fontsize{20pt}{20pt}\selectfont Email]\hspace*{40pt} \CJKunderdblline {\fontsize{20pt}{20pt}\selectfont  $xxxxx@xxxxx.com$\hspace*{145pt}}
    \item[\kaishu \fontsize{20pt}{20pt}\selectfont 日期]\hspace*{52pt}\CJKunderdblline {\fontsize{20pt}{20pt}\selectfont  xxxx-xx-xx \hspace*{225pt}}
%根据需求动态调整\hspace{}中的占位空格
\end{description}
\setcounter{page}{0}%封面不计页数
\thispagestyle{empty}%封面不设置页眉页脚
\newpage



%目录页开始

\begin{center}
	\tableofcontents
\end{center}



\newpage
%第一个实验
\section{\songti实验\,1}

\subsection{\songti题目:}
\noindent\kaishu \textbf{题目要求}
\begin{description}
	\item [\kaishu输入]
	\begin{enumerate}
		\item \kaishu 输入方式
		\item \kaishu 数据范围
	\end{enumerate}
	\item [\kaishu实现]
	\begin{enumerate}
		\item 实现方法
		\end{enumerate}
	\item [\kaishu输出]
	\begin{enumerate}
		\item []\kaishu 输出内容以及方式
	\end{enumerate}
\end{description}

\subsection{\songti数据设计}
\begin{itemize}
	\item\kaishu类层面上
	\begin{description}
		\item[X\,\,class]\kaishu 整体上存储的数据和实现的功能。
		\item[Y\,\,class]\kaishu 整体上存储的数据和实现的功能。
	\end{description}
	\item\kaishu属性和方法层面上
	\begin{description}
		\item[Solution03\,class]
		\begin{enumerate}
			\item[] \hspace{2pt}
			\item \kaishu $int\,\,n:$\,存储内容和功能。
			\item \kaishu $String[\,]str:$\,存储内容和功能。
			\item \kaishu $Scanner\,scr:$\,存储内容和功能。
		\end{enumerate}
	\end{description}
\end{itemize}


\subsection{\songti算法设计}
\begin{itemize}
	\item \kaishu 算法流程图
	\begin{center}
	%	\includegraphics[scale =xxx]{xxx.png}
	\end{center}
	\item \kaishu 主要思路
	\begin{description}
		\item[\kaishu XXX]
			\begin{enumerate}
		\item[]\hspace{2pt}
				\item\kaishu
				\item\kaishu
				\item\kaishu
			\end{enumerate}
		\item[\kaishu XXX]
			\begin{enumerate}
				\item[]\hspace{2pt}
				\item\kaishu
				\item\kaishu
				\item\kaishu
			\end{enumerate}
	\end{description}
\end{itemize}

\subsection{\songti主干代码说明}
\begin{center}
	\begin{lstlisting}[language = Java]
Code  and Notes        
//Notes and String values should be in English
	\end{lstlisting}
\end{center}


\subsection{\songti运行结果展示}
\begin{center}
	\begin{lstlisting}[language = Java]
Input...     //Test n
	\end{lstlisting}
\end{center}

\begin{center}
\begin{lstlisting}[language = Java]
Output...
\end{lstlisting}
\end{center}
\subsection{\songti总结和收获}
\begin{enumerate}
	\item\kaishu
	\item\kaishu
	\item\kaishu
\end{enumerate}
\newpage
\section{\songti实验\,2}

\subsection{\songti题目:}
\noindent\kaishu \textbf{题目要求}
\begin{description}
	\item [\kaishu输入]
	\begin{enumerate}
		\item \kaishu 输入方式
		\item \kaishu 数据范围
	\end{enumerate}
	\item [\kaishu实现]
	\begin{enumerate}
		\item 实现方法
		\end{enumerate}
	\item [\kaishu输出]
	\begin{enumerate}
		\item []\kaishu 输出内容以及方式
	\end{enumerate}
\end{description}

\subsection{\songti数据设计}
\begin{itemize}
	\item\kaishu类层面上
	\begin{description}
		\item[X\,\,class]\kaishu 整体上存储的数据和实现的功能。
		\item[Y\,\,class]\kaishu 整体上存储的数据和实现的功能。
	\end{description}
	\item\kaishu属性和方法层面上
	\begin{description}
		\item[Solution03\,class]
		\begin{enumerate}
			\item[] \hspace{2pt}
			\item \kaishu $int\,\,n:$\,存储内容和功能。
			\item \kaishu $String[\,]str:$\,存储内容和功能。
			\item \kaishu $Scanner\,scr:$\,存储内容和功能。
		\end{enumerate}
	\end{description}
\end{itemize}


\subsection{\songti算法设计}
\begin{itemize}
	\item \kaishu 算法流程图
	\begin{center}
	%	\includegraphics[scale =xxx]{xxx.png}
	\end{center}
	\item \kaishu 主要思路
	\begin{description}
		\item[\kaishu XXX]
			\begin{enumerate}
		\item[]\hspace{2pt}
				\item\kaishu
				\item\kaishu
				\item\kaishu
			\end{enumerate}
		\item[\kaishu XXX]
			\begin{enumerate}
				\item[]\hspace{2pt}
				\item\kaishu
				\item\kaishu
				\item\kaishu
			\end{enumerate}
	\end{description}
\end{itemize}

\subsection{\songti主干代码说明}
\begin{center}
	\begin{lstlisting}[language = Java]
Code  and Notes        
//Notes and String values should be in English
	\end{lstlisting}
\end{center}


\subsection{\songti运行结果展示}
\begin{center}
	\begin{lstlisting}[language = Java]
Input...     //Test n
	\end{lstlisting}
\end{center}

\begin{center}
\begin{lstlisting}[language = Java]
Output...
\end{lstlisting}
\end{center}
\subsection{\songti总结和收获}
\begin{enumerate}
	\item\kaishu
	\item\kaishu
	\item\kaishu
\end{enumerate}
\newpage


\section{\songti实验\,3}

\subsection{\songti题目:}
\noindent\kaishu \textbf{题目要求}
\begin{description}
	\item [\kaishu输入]
	\begin{enumerate}
		\item \kaishu 输入方式
		\item \kaishu 数据范围
	\end{enumerate}
	\item [\kaishu实现]
	\begin{enumerate}
		\item 实现方法
		\end{enumerate}
	\item [\kaishu输出]
	\begin{enumerate}
		\item []\kaishu 输出内容以及方式
	\end{enumerate}
\end{description}

\subsection{\songti数据设计}
\begin{itemize}
	\item\kaishu类层面上
	\begin{description}
		\item[X\,\,class]\kaishu 整体上存储的数据和实现的功能。
		\item[Y\,\,class]\kaishu 整体上存储的数据和实现的功能。
	\end{description}
	\item\kaishu属性和方法层面上
	\begin{description}
		\item[Solution03\,class]
		\begin{enumerate}
			\item[] \hspace{2pt}
			\item \kaishu $int\,\,n:$\,存储内容和功能。
			\item \kaishu $String[\,]str:$\,存储内容和功能。
			\item \kaishu $Scanner\,scr:$\,存储内容和功能。
		\end{enumerate}
	\end{description}
\end{itemize}


\subsection{\songti算法设计}
\begin{itemize}
	\item \kaishu 算法流程图
	\begin{center}
	%	\includegraphics[scale =xxx]{xxx.png}
	\end{center}
	\item \kaishu 主要思路
	\begin{description}
		\item[\kaishu XXX]
			\begin{enumerate}
		\item[]\hspace{2pt}
				\item\kaishu
				\item\kaishu
				\item\kaishu
			\end{enumerate}
		\item[\kaishu XXX]
			\begin{enumerate}
				\item[]\hspace{2pt}
				\item\kaishu
				\item\kaishu
				\item\kaishu
			\end{enumerate}
	\end{description}
\end{itemize}

\subsection{\songti主干代码说明}
\begin{center}
	\begin{lstlisting}[language = Java]
Code  and Notes        
//Notes and String values should be in English
	\end{lstlisting}
\end{center}


\subsection{\songti运行结果展示}
\begin{center}
	\begin{lstlisting}[language = Java]
Input...     //Test n
	\end{lstlisting}
\end{center}

\begin{center}
\begin{lstlisting}[language = Java]
Output...
\end{lstlisting}
\end{center}
\subsection{\songti总结和收获}
\begin{enumerate}
	\item\kaishu
	\item\kaishu
	\item\kaishu
\end{enumerate}
\newpage


\section{\songti实验\,4}

\subsection{\songti题目:}
\noindent\kaishu \textbf{题目要求}
\begin{description}
	\item [\kaishu输入]
	\begin{enumerate}
		\item \kaishu 输入方式
		\item \kaishu 数据范围
	\end{enumerate}
	\item [\kaishu实现]
	\begin{enumerate}
		\item 实现方法
		\end{enumerate}
	\item [\kaishu输出]
	\begin{enumerate}
		\item []\kaishu 输出内容以及方式
	\end{enumerate}
\end{description}

\subsection{\songti数据设计}
\begin{itemize}
	\item\kaishu类层面上
	\begin{description}
		\item[X\,\,class]\kaishu 整体上存储的数据和实现的功能。
		\item[Y\,\,class]\kaishu 整体上存储的数据和实现的功能。
	\end{description}
	\item\kaishu属性和方法层面上
	\begin{description}
		\item[Solution03\,class]
		\begin{enumerate}
			\item[] \hspace{2pt}
			\item \kaishu $int\,\,n:$\,存储内容和功能。
			\item \kaishu $String[\,]str:$\,存储内容和功能。
			\item \kaishu $Scanner\,scr:$\,存储内容和功能。
		\end{enumerate}
	\end{description}
\end{itemize}


\subsection{\songti算法设计}
\begin{itemize}
	\item \kaishu 算法流程图
	\begin{center}
	%	\includegraphics[scale =xxx]{xxx.png}
	\end{center}
	\item \kaishu 主要思路
	\begin{description}
		\item[\kaishu XXX]
			\begin{enumerate}
		\item[]\hspace{2pt}
				\item\kaishu
				\item\kaishu
				\item\kaishu
			\end{enumerate}
		\item[\kaishu XXX]
			\begin{enumerate}
				\item[]\hspace{2pt}
				\item\kaishu
				\item\kaishu
				\item\kaishu
			\end{enumerate}
	\end{description}
\end{itemize}

\subsection{\songti主干代码说明}
\begin{center}
	\begin{lstlisting}[language = Java]
Code  and Notes        
//Notes and String values should be in English
	\end{lstlisting}
\end{center}


\subsection{\songti运行结果展示}
\begin{center}
	\begin{lstlisting}[language = Java]
Input...     //Test n
	\end{lstlisting}
\end{center}

\begin{center}
\begin{lstlisting}[language = Java]
Output...
\end{lstlisting}
\end{center}
\subsection{\songti总结和收获}
\begin{enumerate}
	\item\kaishu
	\item\kaishu
	\item\kaishu
\end{enumerate}
\newpage


\section{\songti实验\,5}

\subsection{\songti题目:}
\noindent\kaishu \textbf{题目要求}
\begin{description}
	\item [\kaishu输入]
	\begin{enumerate}
		\item \kaishu 输入方式
		\item \kaishu 数据范围
	\end{enumerate}
	\item [\kaishu实现]
	\begin{enumerate}
		\item 实现方法
		\end{enumerate}
	\item [\kaishu输出]
	\begin{enumerate}
		\item []\kaishu 输出内容以及方式
	\end{enumerate}
\end{description}

\subsection{\songti数据设计}
\begin{itemize}
	\item\kaishu类层面上
	\begin{description}
		\item[X\,\,class]\kaishu 整体上存储的数据和实现的功能。
		\item[Y\,\,class]\kaishu 整体上存储的数据和实现的功能。
	\end{description}
	\item\kaishu属性和方法层面上
	\begin{description}
		\item[Solution03\,class]
		\begin{enumerate}
			\item[] \hspace{2pt}
			\item \kaishu $int\,\,n:$\,存储内容和功能。
			\item \kaishu $String[\,]str:$\,存储内容和功能。
			\item \kaishu $Scanner\,scr:$\,存储内容和功能。
		\end{enumerate}
	\end{description}
\end{itemize}


\subsection{\songti算法设计}
\begin{itemize}
	\item \kaishu 算法流程图
	\begin{center}
	%	\includegraphics[scale =xxx]{xxx.png}
	\end{center}
	\item \kaishu 主要思路
	\begin{description}
		\item[\kaishu XXX]
			\begin{enumerate}
		\item[]\hspace{2pt}
				\item\kaishu
				\item\kaishu
				\item\kaishu
			\end{enumerate}
		\item[\kaishu XXX]
			\begin{enumerate}
				\item[]\hspace{2pt}
				\item\kaishu
				\item\kaishu
				\item\kaishu
			\end{enumerate}
	\end{description}
\end{itemize}

\subsection{\songti主干代码说明}
\begin{center}
	\begin{lstlisting}[language = Java]
Code  and Notes        
//Notes and String values should be in English
	\end{lstlisting}
\end{center}


\subsection{\songti运行结果展示}
\begin{center}
	\begin{lstlisting}[language = Java]
Input...     //Test n
	\end{lstlisting}
\end{center}

\begin{center}
\begin{lstlisting}[language = Java]
Output...
\end{lstlisting}
\end{center}
\subsection{\songti总结和收获}
\begin{enumerate}
	\item\kaishu
	\item\kaishu
	\item\kaishu
\end{enumerate}
\newpage

\section{\songti实验\,6}

\subsection{\songti题目:}
\noindent\kaishu \textbf{题目要求}
\begin{description}
	\item [\kaishu输入]
	\begin{enumerate}
		\item \kaishu 输入方式
		\item \kaishu 数据范围
	\end{enumerate}
	\item [\kaishu实现]
	\begin{enumerate}
		\item 实现方法
		\end{enumerate}
	\item [\kaishu输出]
	\begin{enumerate}
		\item []\kaishu 输出内容以及方式
	\end{enumerate}
\end{description}

\subsection{\songti数据设计}
\begin{itemize}
	\item\kaishu类层面上
	\begin{description}
		\item[X\,\,class]\kaishu 整体上存储的数据和实现的功能。
		\item[Y\,\,class]\kaishu 整体上存储的数据和实现的功能。
	\end{description}
	\item\kaishu属性和方法层面上
	\begin{description}
		\item[Solution03\,class]
		\begin{enumerate}
			\item[] \hspace{2pt}
			\item \kaishu $int\,\,n:$\,存储内容和功能。
			\item \kaishu $String[\,]str:$\,存储内容和功能。
			\item \kaishu $Scanner\,scr:$\,存储内容和功能。
		\end{enumerate}
	\end{description}
\end{itemize}


\subsection{\songti算法设计}
\begin{itemize}
	\item \kaishu 算法流程图
	\begin{center}
	%	\includegraphics[scale =xxx]{xxx.png}
	\end{center}
	\item \kaishu 主要思路
	\begin{description}
		\item[\kaishu XXX]
			\begin{enumerate}
		\item[]\hspace{2pt}
				\item\kaishu
				\item\kaishu
				\item\kaishu
			\end{enumerate}
		\item[\kaishu XXX]
			\begin{enumerate}
				\item[]\hspace{2pt}
				\item\kaishu
				\item\kaishu
				\item\kaishu
			\end{enumerate}
	\end{description}
\end{itemize}

\subsection{\songti主干代码说明}
\begin{center}
	\begin{lstlisting}[language = Java]
Code  and Notes        
//Notes and String values should be in English
	\end{lstlisting}
\end{center}


\subsection{\songti运行结果展示}
\begin{center}
	\begin{lstlisting}[language = Java]
Input...     //Test n
	\end{lstlisting}
\end{center}

\begin{center}
\begin{lstlisting}[language = Java]
Output...
\end{lstlisting}
\end{center}
\subsection{\songti总结和收获}
\begin{enumerate}
	\item\kaishu
	\item\kaishu
	\item\kaishu
\end{enumerate}
\newpage

\section{\songti源码}
\subsection{\heiti实验\,1}
\begin{center}
\begin{lstlisting}[language=Java]
CodeBlocks
\end{lstlisting}
\end{center}
\subsection{\heiti实验\,2}
\begin{center}
\begin{lstlisting}[language=Java]
CodeBlocks
\end{lstlisting}
\end{center}
\subsection{\heiti实验\,3}
\begin{center}
\begin{lstlisting}[language=Java]
CodeBlocks
\end{lstlisting}
\end{center}
\subsection{\heiti实验\,4}
\begin{center}
\begin{lstlisting}[language=Java]
CodeBlocks
\end{lstlisting}
\end{center}
\subsection{\heiti实验\,5}
\begin{center}
\begin{lstlisting}[language=Java]
CodeBlocks
\end{lstlisting}
\end{center}
\subsection{\heiti实验\,6}
\begin{center}
\begin{lstlisting}[language=Java]
CodeBlocks
\end{lstlisting}
\end{center}




\end{document}